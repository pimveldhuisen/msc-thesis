\chapter{Prior Work}
\section{Bitcoin}
A recent system that has managed to maintain a consistent global state across multiple untrusted nodes is Bitcoin (and so have many of its derivatives). Here the global state consists of a number of accounts with different amounts of Bitcoins in each. This global state is however not achieved in a truly distributed way; instead, Bitcoin uses temporary centrality. This means that one of the nodes in the network will be picked to create an increment of the global state. This increment is known as a block, and consists of a combination of smaller increments, called transactions. These transactions are spread in a peer-to-peer fashion, but the node that creates the block decides which will be included in the new global state. The privilige of temporarily being the central node is aquired by a Proof-of-work principle. This means that all nodes will attempt to solve cryptographivcal puzzles, and those who succeed become the central node, which comes with some reward (miners fee). When different nodes disagree of the global state, a fork is created. However, the chain with the larges amount of blocks will be considerd as the source of truth, and thus only these blocks will contribute to the global state. The idea behind this is that a large amount of computational power is required to change the global state. This amount of power is available among those who strive to achieve the common goal of the network, but an indivual whose strive to acievhe his own indiviual goals can not allocate enough computational power to do this. 
