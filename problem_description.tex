\chapter{Problem Description}
The goal of this thesis is to implement the ideas of the multichain in a real world scenario and provide experimental validation of the concepts and operational principles of the system. The ultimate objective the multichain is designed for is to enforce cooperation of peers in a decentralized manner. In order to achieve this, the peers should be incentiveced is such a way that their goals are aligned with the goal of the network. Behavior that is benefical to the individual peer should also be benefical for the network as a whole. In concrete, this means that uploading to the network should be rewarded appropriately.\\
\\
To reward certain behavior in a peer to peer network, other peers must be made aware of this behavior. We therefore need some sort of record keeping system that keeps track of what peers do. In the network, behavior consists of interactions between peers. A simple record keeping system would consist of each peer maintaining a list of his own interactions. Other peers could then receive the list, and decide based on these interactions whether the peer has shown good behavior in the past and should be rewarded or not. While this approach does in theory provides some incentives to be cooperative, it is obviously very tempting to cheat by providing a modified list, that shows fake good behavior. The goal of the multichain is to prevent this kind of cheating by making it infeasible, thus providing a tamper-proof record keeping system.\\
\\
By design, the multichain is a record keeping system only; it does not in itself define good behavior or the rewards that come with such behavior. It merely keeps track of interactions that have taken place in the network. To prevent free riding and provide the right incentives to contribute to the network another system will have to work on top of the multichain. The systems then form a multilayered software architecture with some of the benefits that come from such an architecture. It is possible for different peers in the network to have different policies for judging and rewarding behavior, thus making different decicions based on the same multichain. The layers are however not completely independent, as some policies in the higher layer might affect the feasibility of certain ways of cheating.\\
\\
- Distributed	
autonomous, decentral.
churn resilient 

\\
- Scalability 
store info on thousanbds of users locally, 
\\
implement this + real world testing experimental proof
\\

